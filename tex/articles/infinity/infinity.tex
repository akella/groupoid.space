\documentclass{aip-cp}
\usepackage{listings}
\usepackage[numbers]{natbib}
\usepackage[only,llbracket,rrbracket,llparenthesis,rrparenthesis]{stmaryrd}
\usepackage{graphicx}
\usepackage{amsmath}
\usepackage{amssymb}
\usepackage{txfonts}
\usepackage[utf8]{inputenc}
\usepackage{url}

\begin{document}
\title{Pure Type Systems to Homotopy Calculus:\\ The Infinity Language Stack}
\author[aff1]{Maksym Sokhatskyi\corref{cor1}}
\eaddress[url]{https://groupoid.space}
\author[aff1]{Pavlo Maslianko\corref{cor3}}

\affil[aff1]{National Technical University of Ukraine ``Igor Sikorsky Kyiv Polytechnical Institute''}
\corresp[cor1]{maxim@synrc.com}
\corresp[cor3]{mppdom@i.ua}

\maketitle
\begin{abstract}
We expose the layered models of type systems with the arising complexity.
While designing the language or type checker it is useful to do a separation or reusing of AST trees.
In the resulting tower we have four languages:
1) PTS\footnote{Pure Type System, Henk Barendregt} \cite{Erik97} with infinite number of universes;
2) MLTT\footnote{Martin-Löf Type Theory, as for 1984} \cite{Lof84} system with Pi, Sigma and Equ;
3) Calculus of Inductive Constructions \cite{Mohring15} --- system with induction principle using backends as
encodings: IR-types\footnote{Induction-Recursion} \cite{Dagand13} and W-types\footnote{Wellfounded Trees} backends;
4) Homotopy Calculus with interval [0,1] \cite{Mortberg17}. The models are given in cubical type checker,
which is CCHM-fibrations\footnote{Cohen-Coquand-Huber-Mörtberg fibrations} \cite{Orton17} based type checker.
The central part and main motivation in type theory is Equality type that is fully
derivable with its eliminators and computational rules in CCHM model. We show examples of
Path Equality from a future base library and give some insights on future development.
\end{abstract}

\section{Introduction}
The idea of Infinity Langauge came from the needs of unification and
arranging different calculuses as extensions to the core of the language
with dependent types (or MLTT core). E.g. pi-calculus (spawn, send, recv, pub, sub)
and stream calculus (map, scan, concat, reduce, iota, fold, split) are connected
as streams represent variables in pi-calculus. This gives us a direction to build
a solution towards unified layered type checker.

This type checker should introduce new complexity on each layer while
remaining compatible with properties on previous layer. Here is shown:
1) Pure Type System with infinite number of universes for consistency as a base calculus;
2) Martin-Löf Type Theory as intermediate calculus with sigma and identity reasoning for pattern matching;
3) EXE calculus with inductive types, inductively recursive IR-encoding and W-types encoding and derivable induction principle, therefore fixpoint type.
4) Homotopy calculus based on interval $[0,1]$, its introduction, eliminators, composition and gluening.

\begin{table}[h]
\caption{Type Systems}
\label{tab:a}
\tabcolsep7pt\begin{tabular}{lcccc}
\hline
\tch{1}{c}{b}{PTS} & \tch{1}{c}{b}{MLTT} & \tch{1}{c}{b}{CiC} & \tch{1}{c}{b}{CCHM}\\
\hline
PI        & PI    & PI    & PI \\
          & SIGMA & SIGMA & SIGMA \\
          & ID\footnote{as for MLTT-84.}   & ID    & PATH \\
          &       & INDUCTION   & COMPOSITION \\
          &       &       & GLUENING \\
\hline
\end{tabular}
\end{table}

\section{Pure Type Systems}

From the time Coquand at all discovered Calculus of Constructions \cite{Coq88},
and Barendregt \cite{Henk93} systemized its variations, a Pure Type System
theory was developed. It is known also as Single Axiom System
with only Pi-Type of MLTT \cite{Lof84}, representing functional completeness.

\begin{lstlisting}[mathescape=true]
data pts
    = star             (n: nat)
    | var    (x: name) (l: nat)
    | pi     (x: name) (l: nat) (d c: lang)
    | lambda (x: name) (l: nat) (d c: lang)
    | app                       (f a: lang)
\end{lstlisting}

This language is called one axiom language (or pure) as eliminator
and introduction rules inferred from type formation rule.
The only computation rule of Pi type is called beta-reduction.

\begin{equation}
\tag{$\Pi$-formation}
\dfrac
{A : Type \vdash B : A \rightarrow Type}
{\Pi\ (x:A) \rightarrow B(x) : Type}
\end{equation}

\begin{equation}
\tag{$\lambda$-intro}
\dfrac
{x:A \vdash b : B}
{\lambda\ (x:A) \rightarrow b : \Pi\ (x:A) \rightarrow B(x) }
\end{equation}

\begin{equation}
\tag{$App$-elimination}
\dfrac
{f: (\Pi\ (x:A) \rightarrow B x)\ \ \ a: A}
{f\ a : B x\ [a/x]}
\end{equation}

\begin{equation}
\tag{$\beta$-computation}
\dfrac
{x:A \vdash b: B\ \ \ a:A}
{(\lambda\ (x:A) \rightarrow b)\ a = b\ [a/x] : B\ [a/x]}
\end{equation}

\begin{equation}
\tag{subst}
\dfrac
{\pi_1 : A\ \ \ \ u:A \vdash \pi_2 : B}
{[\pi_1/u]\ \pi_2 : B}
\end{equation}

\section{Sigma Types}

The basic Core primitive which is needed for proving things
is MLTT Sigma-Type, representing Contextual Completeness.
This is needed for building Sigma pairs which are curried
records. Usually, type checkers called Pi-Sigma provers
as it contains PTS enriched with Sigma primitive.

\begin{lstlisting}[mathescape=true]
data exists
    = sigma (n: name) (a b: lang)
    | pair (a b: lang)
    | fst (p: lang)
    | snd (p: lang)
\end{lstlisting}

\begin{equation}
\tag{$\Sigma$-formation}
\dfrac
{x:A \vdash B : Type}
{\Sigma\ (x:A) * B x : Type}
\end{equation}

\begin{equation}
\tag{$\Sigma$-intro}
\dfrac
{x:A \vdash b : B(x), a: A}
{(a,b) : \Sigma\ (x:A) * B(x) }
\end{equation}

\begin{equation}
\tag{$\pi_1$-elimination}
\dfrac
{A: type, B: A -> Type}
{\pi_1 : \Sigma\ (x:A) * B x \rightarrow A}
\end{equation}

\begin{equation}
\tag{$\pi_1$-elimination}
\dfrac
{A: type, B: A -> Type}
{\pi_2 : \Sigma\ (x:A) * B x \rightarrow B x}
\end{equation}

\newpage
E.g. one may want to define vectors as refinement type of list type using sigma pair construction:

\begin{lstlisting}[mathescape=true]
    vector (n: nat) (A: U): U
       = (c: list nat)
       * (Path nat (length nat c) n)
\end{lstlisting}

instead of using initial object with restriction in second parameter of vcons constructor:

\begin{lstlisting}[mathescape=true]
    data vector (A: U) (n: nat)
       = vnil
       | vcons (x: A) (xs: vector A (pred n))
\end{lstlisting}

We created an PTS type checker for Erlang/OTP platform \footnote{https://github.com/groupoid/om}.
The download and run instructions could be obtained from Github page.


\section{Identity Types}

Starting from an appearance of MLTT systems there was a question about derivability of equality
along with J eliminator and functional extensionality. While IR-encoding \cite{Dagand13} provides
the useful descriptive mechanism for handling equality eliminators, homotopical interval
as a model of equality is encoded internally within CCHM-fibrations system \cite{Orton17}. The
identity type could also be modeled using setoids \cite{Bishop67}, but it will lacks functional
extensionality which is also non-derivable in MLTT systems.

\begin{lstlisting}[mathescape=true]
data identity
    = id (a b: lang)
    | idpair (a b: lang)
    | idelim (a b pa pb u: lang)
\end{lstlisting}

The core of elimination rule is a substitution or transport that gives as
a fact of equality of points in dependent types indexed by two points with a given equality.

\begin{equation}
\tag{$Id$-formation}
\dfrac
  {a:A\ \ \ \ b:A\ \ \ \ A:Type}
  {Id(A,a,b) : Type}
\end{equation}

\begin{equation}
\tag{$Id$-intro}
\dfrac
  {a:A}
  {refl(A,a) : Id(A,a,a) }
\end{equation}

\begin{equation}
\tag{$J$-elimination}
\dfrac
  {p:Id(a,b)\ \ \ \ x,y:A\ \ \ \ u:Id(x,y) \vdash E:Type\ \ \ \ x:A \vdash d: E\ [x/y,\ refl(x)/u]}
  {J(a,b,p,(x,y,u)\ d) : E\ [a/x,\ b/y,\ p/u]}
\end{equation}

\begin{equation}
\tag{$Id$-computation}
\dfrac
  {a,x,y:A,\ \ \ \ u:Id(x,y) \vdash E:Type\ \ \ \ x:A \vdash d:E\ [x/y,\ refl(x)/u]}
  {J(a,a,refl(a),(x,y,u)\ d) = d\ [a/x] : E\ [a/y,\ refl(a)/u]}
\end{equation}

The elimination rule of indentity type cannot be expressed inside MLTT. That is why in all provers
before CCHM-based cubical have encoded J eliminator as part of AST and core language primitive.
However lately we where able to model a theory where J is derivable using $[0,1]$
homotopy interval \cite{Mortberg17} and CCHM-fibrations. Here is example of Identity Type theory
using cubical type checker:

\begin{lstlisting}[mathescape=true]
Path   (A: U) (a b: A): U = PathP (<i> A) a b
refl   (A: U) (a: A): Path A a a = <i> a
singl  (A: U) (a: A): U = (x : A) * Path A a x
sym    (A: U) (a b: A) (p: Path A a b): Path A b a = <i> p @ -i
trans  (A B: U) (p: Path U A B) (a: A): B = comp p a []
\end{lstlisting}

\newpage
The differences between congruence, functional extensionality,
substitution and contractablity of singletons are given in explicit manner
by their cubical definitions.

\begin{lstlisting}[mathescape=true]
cong   (A B: U) (f: A -> B) (a b: A) (p: Path A a b):
    Path B (f a) (f b) = <i> f (p @ i)

funExt (A B: U) (f g: A -> B) (p: (x: A) -> Path B (f x) (g x)):
    Path (A -> B) f g = <i> \(a : A) -> p a @ i

mapOnPath (A B: U) (f: A -> B) (a b: A) (p: Path A a b):
    Path B (f a) (f b) = <i> f (p @ i)

subst (A: U) (P: A -> U) (a b: A) (p: Path A a b) (e: P a):
    P b = comp (<i> P (p @ i)) e []

contrSingl (A: U) (a b: A) (p: Path A a b):
    Path (singl A a) (a,refl A a) (b,p) = <i> (p @ i,<j> p @ i/\j)
\end{lstlisting}

\subsection*{Identity Eliminator J}

J is formulated in a form of Paulin-Mohring and implemented
using two facts that singleton are contractible and dependent function transport.

\begin{lstlisting}[mathescape=true]
JPM (A: U) (a b: A) (P: singl A a -> U) (u: P (a,refl A a)) (p: Path A a b):
    P (b,p) = subst (singl A a) T (a,refl A a) (b,p) (contrSingl A a b p) u
    where T (z : singl A a) : U = P z
\end{lstlisting}

Another J based on path induction from HoTT book \cite{HoTT13}.

\begin{lstlisting}[mathescape=true]
J (A: U) (a: A) (C: (x : A) -> Path A a x -> U)
  (d: C a (refl A a)) (x : A) (p: Path A a x) : C x p =
    subst (singl A a) T (a, refl A a) (x, p) (contrSingl A a x p) d
      where T (z: singl A a): U = C (z.1) (z.2)
\end{lstlisting}

\begin{lstlisting}[mathescape=true]
composition (A: U) (a b c: A) (p: Path A a b) (q: Path A b c): Path A a c
    = <i> comp (<j>A) (p @ i) [ (i = 1) -> q, (i=0) -> <j> a ]
\end{lstlisting}

\section{Inductive Types}

The further development of induction \cite{Dybjer94,Vene00} inside MLTT provers led
to the theory of polynomial functors and well-founded trees \cite{Gambino03},
known in programming languages as inductive types with data
and record core primitives of type checker. In fact recursive inductive
types \cite{Wadler90} could be encoded in PTS using non-recursive representation
Berarducci \cite {Bohm85} or categorical encoding.

\begin{lstlisting}[mathescape=true]
data tele (A: U)   = emp | tel (n: name) (b: A) (t: tele A)
data branch (A: U) =        br (n: name) (args: list name) (term: A)
data label (A: U)  =       lab (n: name) (t: tele A)

data ind
    = data_  (n: name) (t: tele lang) (labels:   list (label lang))
    | case   (n: name) (t: lang)      (branches: list (branch lang))
    | ctor   (n: name)                (args:     list lang)
\end{lstlisting}

\begin{equation}
\tag{$W$-formation}
\dfrac
{A:Type\ \ \ \ x:A\ \ \ \ B(x):Type}
{W (x:A) \rightarrow B(x) : Type}
\end{equation}

\begin{equation}
\tag{$W$-intro}
\dfrac
{a:A\ \ \ \ t: B(a) \rightarrow W}
{sup(a,t) : W}
\end{equation}

\begin{equation}
\tag{$W$-elimination}
\dfrac
{w: W \vdash C(w) : Type\ \ \ \ x:A,\ u:B(x) \rightarrow W,\ \ \ \ v:\Pi (y:B(x)) \rightarrow C(u(y)) \vdash c(x,u,v):C(sup(x,u))}
{w:W \vdash wrec(w,c):C(w)}
\end{equation}

\begin{equation}
\tag{$W$-computation}
\dfrac
{w: W \vdash C(w) : Type\ \ \ \ x:A,\ u:B(x) \rightarrow W,\ \ \ \ v:\Pi (y:B(x)) \rightarrow C(u(y)) \vdash c(x,u,v):C(sup(x,u))}
{x:A,\ \ \ \ u:B(x) \rightarrow W \vdash wrec(sup(x,u),c)\ =\ c(x,u,\lambda (y:B(x)) \rightarrow wrec(u(y),c)):C(sup(x,u))}
\end{equation}

The non-well-founded trees or infinite coinductive trees \cite{Jacobs97,Basold16}
are useful for modeling infinite process running which is
part of Milner's Pi-calculus. Coinductive streams are
part of any MLTT base library.

We gathered several models in lambek module of base library \footnote{https://github.com/groupoid/infinity/blob/master/priv/lambek.ctt}.
It includes:
1) Recursion Schemes;
2) Catamorphism Inductive Encoding;
3) IR-encoding \cite{Dagand13};
4) F-Algebra encoding  \cite{Hinze13}.

\section{Higher Inductive Types}

The fundamental development of equality inside MLTT
provers led us to the notion of $\infty$-groupoid \cite{Streicher95} as spaces.
In this was Path identity type appeared in the core
of type checker along with de Morgan algebra on
built-in interval type. Glue, unglue composition
and fill operations are also needed in the core
for the univalence computability \cite{Mortberg17}.

\begin{lstlisting}[mathescape=true]
data hts
    = path (a b: lang)
    | path_lam (n: name) (a b: lang)
    | path_app (f: name) (a b: lang)
    | comp_ (a b: lang)
    | fill_ (a b c: lang)
    | glue_ (a b c: lang)
    | glue_elem (a b: lang)
    | unglue_elem (a b: lang)
\end{lstlisting}

\section{Comparing Homotopical and Inductive proofs}

Here we show how to use Path equality from CCHM {\bf cubical} type checker:

\begin{lstlisting}[mathescape=true]
add_comm (a : nat) : (n : nat) -> Path nat (add a n) (add n a) = split
  zero  -> <i> add_zero a @ -i
  suc m -> <i> comp (<_> nat) (suc (add_comm a m @ i))
                    [ (i = 0) -> <j> suc (add a m)
                    , (i = 1) -> <j> add_suc m a @ -j ]
\end{lstlisting}

and how it is differ e.g. with inductive based proofs from {\bf Coq}:

\begin{lstlisting}[mathescape=true]
Theorem plus_comm : forall n m : nat, n + m = m + n.
Proof.
  intros n m.
  induction n as [| n' Sn' ].
  - simpl. rewrite <- plus_n_0. reflexivity.
  - simpl. rewrite <- plus_n_Sm. rewrite <- Sn'. reflexivity.
Qed.
\end{lstlisting}

\section{Conclusion}
In this article we gave a simple AST definitions for tower of languages
on top of Pure Type System, show a cubical terms for Identity types, compare
the inductive and homotopical proof styles. The main motivation in
layering the complex proving languages is to simplify the reasoning about them,
manage and reduce the stack of calculuses.

\begin{lstlisting}[mathescape=true]
data lang
   = PTS (_: pts)
   | SIGMA (_: exists)
   | ID (_: identity)
   | INDUCTION (_: ind)
   | CUBICAL (_: hts)
\end{lstlisting}

Such decoupling of top level language brings modularity to its essence,
the type checker. We can reason about properties of cacluluses polymorphically
disregarding of underlying types in computational equations, such as
$Path\ (B\ a)\ ((app\ B\ (lambda\ B\ f))\ a)\ (f\ a)$. Here we don't know $a$ is
term of PTS or SIGMA language, this equation holds for all terms $a$.
We treating language core as AST sum of sublanguages, that is
extendable to the needed set of language features.

\bibliographystyle{aipnum-cp}
\bibliography{infinity}
\newpage
\tableofcontents

\end{document}
