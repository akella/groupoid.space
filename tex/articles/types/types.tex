% copyright (c) 2018 Groupoid Infinity

\documentclass{article}
\usepackage{amscd}
\usepackage{listings}
\usepackage[numbers]{natbib}
\usepackage[only,llbracket,rrbracket,llparenthesis,rrparenthesis]{stmaryrd}
\usepackage{graphicx}
\usepackage{amsmath}
\usepackage{amssymb}
\usepackage{txfonts}
\usepackage{tikz-cd}
\usepackage[utf8]{inputenc}

\newcommand*{\thead}[1]{\multicolumn{1}{c}{\bfseries #1}}

\begin{document}

\title{Mathematical Components for Cubical}
\author{Maksym Sokhatskyi $^1$ and Pavlo Maslianko $^1$}
\date{
    $^1$ National Technical University of Ukraine ``Igor Sikorsky Kyiv Polytechnical Institute''
    \today
}

\maketitle

\begin{abstract}
Keywords: Formal Methods, Type Theory, Computer Languages,
          Theoretical Computer Science, Applied Mathematics,
          Cubical Type Theory, Martin-Löf Type Theory
\end{abstract}

\section{Intro}

This library is dedicated to cubical-compatible type checkers based
on homotopy interval [0,1] and MLTT as a core. The base library is
founded on top of 5 core modules: proto (composition, id, const),
path (subst, trans, cong, refl, singl, sym), prop, set (isContr, isProp, isSet),
equiv (fiber, eqiuv) and iso (lemIso, isoPath).
This machinery is enough to prove univalence axiom.

(i) The library has rich recursion scheme primitives
in recursion module, while very basic nat, list, stream
functionality. (ii) The very basic theorems are given
in pi, iso\_pi, sigma, iso\_sigma, retract modules.
(iii) The library has category theory theorems from
HoTT book in cat, fun and category modules.
(iv) The library also includes categorical encoding
of dependent types presented in Cwf module.

This library is best to read with HoTT book.

\begin{table}[h]
\centering
\caption{Types Taxonomy}
\label{tab:a}
\tabcolsep7pt
\begin{tabular}{lcccc}
\hline
\thead{NR+ND} & \thead{R+ND} & \thead{NR+D} & \thead{R+D}\\
\hline
unit        & nat    & path    & vector \\
bool        & list   & proto   & fin \\
either      &        & iso     &  \\
maybe       &        & equiv   &  \\
            &        &         &  \\
\hline
\thead{NR*ND} & \thead{R*ND} & \thead{NR*D} & \thead{R*D}\\
\hline
pure        & stream & sigma   & cat  \\
functor     &        & setoid  & prop \\
applicative &        &         & set  \\
monad       &        &         & groupoid \\
\end{tabular}
\end{table}

\section{MLTT}

\subsection{Pi}
\subsection{Sigma}
\subsection{Identity Type}

\begin{lstlisting}[mathescape=true]
Path     (A: U) (a b: A): U =
singl    (A: U) (a: A): U =
refl     (A: U) (a: A): Path A a a =
sym      (A: U) (a b: A) (p: Path A a b): Path A b a =
inv      (A: U) (a b: A) (p: Path A a b): Path A b a =
eta      (A: U) (a: A): singl A a =
contr    (A: U) (a b: A) (p: Path A a b): Path (singl A a) (eta A a) (b,p) =
cong   (A B: U) (f: A->B) (a b: A) (p: Path A a b): Path B (f a) (f b) =
trans  (A B: U) (p: Path U A B) (a : A): B =
subst    (A: U) (P: A->U) (a b: A) (p: Path A a b) (e: P a): P b =
J        (A: U) (a: A) (C: (x: A) -> Path A a x -> U)
      (d: C a (refl A a)) (x: A) (p: Path A a x): C x p
   = subst (singl A a) T (eta A a) (x, p) (contr A a x p) d
           where T (z: singl A a): U = C (z.1) (z.2)
\end{lstlisting}

\newpage
\section{Runtime Types}

\subsection{Empty and Unit}

\begin{lstlisting}[mathescape=true]
data Empty =
data Unit = tt
\end{lstlisting}

\subsection{Bool}

\begin{lstlisting}[mathescape=true]
data bool = false | true
neg_: bool -> bool
or_:  bool -> bool -> bool
and_: bool -> bool -> bool
bool_case (A: U) (f t: A): bool -> A
bool_eq: bool -> bool -> bool
\end{lstlisting}

\subsection{Either and Tuple}

\begin{lstlisting}[mathescape=true]
data or (A B: U) = inl (a: A) | inr (b: B)
data tuple (A B: U) = pair (a: A) (b: B)
\end{lstlisting}

\subsection{Maybe and Nat}

\begin{lstlisting}[mathescape=true]
data maybe (A: U) = nothing | just (a: A)
data nat = zero | succ (n: nat)
\end{lstlisting}

\subsection{List}

\begin{lstlisting}[mathescape=true]
data list (A: U) = nil | cons (a: A) (as: list A)

null (A: U): list A -> bool
head (A: U): list A -> maybe A
tail (A: U): list A -> maybe (list A)
nth (A: U): nat -> list A -> maybe A
append (A: U): list A -> list A -> list A
reverse (A: U): list A -> list A = rev nil where
map (A B: U) (f: A -> B) : list A -> list B = split
zipWith (A B C: U) (f: A -> B -> C): list A -> list B -> list C
zip (A B: U): list A -> list B -> list (tuple A B)
foldr (A B: U) (f: A -> B -> B) (Z: B): list A -> B
foldl (A B: U) (f: B -> A -> B) (Z: B): list A -> B
switch (A: U) (a b: Unit -> list A) : bool -> list A
filter (A: U) (p: A -> bool) : list A -> list A
uncons (A: U): list A -> maybe ((a: A) * (list A))
length (A: U): list A -> nat
list_eq (A: eq): list A.1 -> list A.1 -> bool
\end{lstlisting}

\subsection{Stream}

\begin{lstlisting}[mathescape=true]
data stream (A: U) = cons (x: A) (xs: stream A)

tail (A: U): stream A -> stream A = split cons x xs -> xs
head (A: U): stream A -> A = split cons x xs -> x
fib (a b: nat): stream nat = cons a (fib b (add a b))
seq (start: nat): stream nat = cons start (seq (succ start))
ones:  stream nat = cons one ones
zeros: stream nat = cons zero zeros
nats:  stream nat = seq zero
\end{lstlisting}

\subsection{Vector and Fin}

\begin{lstlisting}[mathescape=true]
data vector (A: U) (n: nat) = vnil | bcons (x: A) (xs: vector A (pred n))
data fin (n: nat) = fzero | fsucc (_: fin (pred n))
\end{lstlisting}

\subsection{IO}
\subsection{IOI}

\newpage
\section{F-Algebras and Recursion Schemes}

  A F-algebra $(\mu F, in)$ is the initial F-algebra if for any F-algebra $(C, \varphi)$
  there exists a unique arrow $\llparenthesis \varphi \rrparenthesis : \mu F \rightarrow C$ where $f = \llparenthesis \varphi \rrparenthesis$
  and is called catamorphism. Similar a F-coalgebra $(\nu F, out)$ is the terminal
  F-coalgebra if for any F-coalgebra $(C, \varphi)$ there exists unique arrow
  $\llbracket \varphi \rrbracket : C \rightarrow \nu F$ where $f =
  \llbracket \varphi \rrbracket$

\begin{center}
\begin{tabular}{lcl}
\begin{tikzcd}
  F\ \mu F \arrow{d}[left]{F\ \llparenthesis \varphi \rrparenthesis} \arrow{r}{in} & \mu F \arrow{d}{\llparenthesis \varphi \rrparenthesis} \\
  F C \arrow{r}{\varphi} & C \end{tikzcd} & & \begin{tikzcd}
  C \arrow{d}[left]{ \llbracket \varphi \rrbracket} \arrow{r}{\phi} & F\ C\arrow{d}{F\ \llbracket \varphi \rrbracket} \\
  \nu F \arrow{r}{out} & F \nu F\end{tikzcd} \\
  \ & \  &\  \\
  $f \circ in = \varphi \circ F\ f \equiv f = \llparenthesis \varphi \rrparenthesis$& &
  $out \circ f = F\ f \circ \varphi \equiv f = \llbracket \varphi \rrbracket$ \\
\end{tabular}
\end{center}

\subsection{Fixpoint and Free Structures}

\begin{lstlisting}[mathescape=true]
data freeF   (F:U->U)(A B:U)= ReturnF (a:A) | BindF(f:F B)
data cofreeF (F:U->U)(A B:U)= CoBindF (a:A) (f: F B)
data free    (F:U->U)(A:U)  = Free    (_:fix(freeF F A))
data cofree  (F:U->U)(A:U)  = CoFree  (_:fix(cofreeF F A))

unfree   (A: U) (F: U -> U): free   F A -> fix (freeF   F A)
  = split Free   a -> a

uncofree (A: U) (F: U -> U): cofree F A -> fix (cofreeF F A)
  = split CoFree a -> a
\end{lstlisting}

\subsection{Catamorphism}

\begin{lstlisting}[mathescape=true]
cata (A: U) (F: U -> U) (X: functor F)
     (alg: F A -> A) (f: fix F): A
   = alg (X.1 (fix F) A (cata A F X alg) (out_ F f))
\end{lstlisting}

\subsection{Anamorphism}

\begin{lstlisting}[mathescape=true]
ana (A: U) (F: U -> U) (X: functor F)
    (coalg: A -> F A) (a: A): fix F
  = Fix (X.1 A (fix F) (ana A F X coalg) (coalg a))
\end{lstlisting}

\subsection{Inductive Types}

\begin{lstlisting}[mathescape=true]
ind (A: U) (F: U -> U): U
  = (in_: F (fix F) -> fix F)
  * (in_rev: fix F -> F (fix F))
  * (fold_: (F A -> A) -> fix F -> A)
  * Unit
\end{lstlisting}

\begin{lstlisting}[mathescape=true]
inductive (F: U -> U) (A: U) (X: functor F): ind A F
  = (in_ F,out_ F,cata A F X,tt)
\end{lstlisting}

\subsection{Coinductive Types}

\begin{lstlisting}[mathescape=true]
coind (A: U) (F: U -> U): U
  = (out_: fix F -> F (fix F))
  * (out_rev: F (fix F) -> fix F)
  * (unfold_: (A -> F A) -> A -> fix F)
  * Unit
\end{lstlisting}

\begin{lstlisting}[mathescape=true]
coinductive (F: U -> U) (A: U) (X: functor F): coind A F
  = (out_ F,in_ F,ana A F X,tt)
\end{lstlisting}

\newpage
\section{Algebraic Structures}

\begin{lstlisting}[mathescape=true]
isAssociative (M: U) (op: M -> M -> M) : U

hasIdentity (M : U) (op : M -> M -> M) (id : M) : U
  = (_ : hasLeftIdentity M op id)
  * (hasRightIdentity M op id)
\end{lstlisting}

\begin{lstlisting}[mathescape=true]
isMonoid (M: SET): U
  = (op: M.1 -> M.1 -> M.1)
  * (_: isAssociative M.1 op)
  * (id: M.1)
  * (hasIdentity M.1 op id)
\end{lstlisting}

\begin{lstlisting}[mathescape=true]
isCMonoid (M: SET): U
  = (m: isMonoid M)
  * (isCommutative M.1 m.1)
\end{lstlisting}

\begin{lstlisting}[mathescape=true]
isGroup (G: SET): U
  = (m: isMonoid G)
  * (inv: G.1 -> G.1)
  * (hasInverse G.1 m.1 m.2.2.1 inv)
\end{lstlisting}

\begin{lstlisting}[mathescape=true]
isAbGroup (G: SET): U
  = (g: isGroup G)
  * (isCommutative G.1 g.1.1)
\end{lstlisting}

\begin{lstlisting}[mathescape=true]
isRing (R: SET): U
  = (mul: isMonoid R)
  * (add: isAbGroup R)
  * (isDistributive R.1 add.1.1.1 mul.1)
\end{lstlisting}

\begin{lstlisting}[mathescape=true]
isAbRing (R: SET): U
  = (mul: isCMonoid R)
  * (add: isAbGroup R)
  * (isDistributive R.1 add.1.1.1 mul.1.1)
\end{lstlisting}

\newpage
\section{Category Theory}

\begin{lstlisting}[mathescape=true]
isAbRing (R: SET): U
  = (mul: isCMonoid R)
  * (add: isAbGroup R)
  * (isDistributive R.1 add.1.1.1 mul.1.1)
\end{lstlisting}

\subsection{Precategory}

\begin{lstlisting}[mathescape=true]
cat: U = (A: U) * (A -> A -> U)

isPrecategory (C: cat): U
  = (id:      (x: C.1) -> C.2 x x)
  * (c:       (x y z:C.1)->C.2 x y->C.2 y z->C.2 x z)
  * (homSet:  (x y: C.1) -> isSet (C.2 x y))
  * (left:    (x y: C.1) -> (f: C.2 x y) ->
              Path (C.2 x y) (c x x y (id x) f) f)
  * (right:   (x y: C.1) -> (f: C.2 x y) ->
              Path (C.2 x y) (c x y y f (id y)) f)
  * (compose: (x y z w: C.1) -> (f: C.2 x y) ->
              (g: C.2 y z) -> (h: C.2 z w) ->
              Path (C.2 x w) (c x z w (c x y z f g) h)
              (c x y w f (c y z w g h))) * Unit
\end{lstlisting}

\begin{lstlisting}[mathescape=true]
carrier (C: precategory): U = C.1.1
hom     (C: precategory) (a b: carrier C): U = C.1.2 a b
path    (C: precategory) (x: carrier C): hom C x x = C.2.1 x
compose (C: precategory) (x y z: carrier C)
        (f: hom C x y) (g: hom C y z): hom C x z
        = C.2.2.1 x y z f g
\end{lstlisting}

\subsection{Terminal and Initial Objects}

\begin{lstlisting}[mathescape=true]
isInitial (C: precategory) (x: carrier C): U
  = (y: carrier C) -> isContr (hom C x y)
\end{lstlisting}

\begin{lstlisting}[mathescape=true]
isTerminal (C: precategory) (y: carrier C): U
  = (x: carrier C) -> isContr (hom C x y)
\end{lstlisting}

\begin{lstlisting}[mathescape=true]
initialObject (C: precategory): U
  = (x: carrier C)
  * isInitial C x
\end{lstlisting}

\begin{lstlisting}[mathescape=true]
terminalObject (C: precategory): U
  = (y: carrier C)
  * isTerminal C y
\end{lstlisting}

\subsection{Functor}

\begin{lstlisting}[mathescape=true]
catfunctor (A B: precategory): U
  = (ob:   carrier A -> carrier B)
  * (mor:  (x y: carrier A) ->
           hom A x y -> hom B (ob x) (ob y))
  * (id:   (x: carrier A) ->
           Path (hom B (ob x) (ob x))
                (mor x x (path A x)) (path B (ob x)))
  * (cmp:  (x y z: carrier A) -> (f: hom A x y) -> (g: hom A y z) ->
           Path (hom B (ob x) (ob z)) (mor x z (compose A x y z f g))
           (compose B (ob x) (ob y) (ob z) (mor x y f) (mor y z g))) * Unit
\end{lstlisting}

\section{Proto}

\begin{lstlisting}[mathescape=true]
case   (A B C: U)(b:A->C)(c:B->C):or A B->C = split{inl x->b(x);inr y->c(y)}
fst      (A B: U): tuple A B -> A = split pair a b -> a
snd      (A B: U): tuple A B -> B = split pair a b -> b
idfun      (A: U) (a: A): A = a
constfun   (A B: U) (a: B): A -> B = \(_:A) -> a
o      (A B C: U) (f: B -> C) (g: A -> B): A -> C = \(x:A) -> f (g x)
and      (A B: U): U = (_:A) * B
\end{lstlisting}

\begin{lstlisting}[mathescape=true]
efq        (A: U): Empty -> A = split {}
neg        (A: U): U = A -> Empty
dneg  (A:U) (a:A): neg (neg A) = \(h: neg A) -> h a
dec        (A: U): U = or A (neg A)
stable     (A: U): U = neg (neg A) -> A
\end{lstlisting}


\end{document}