\addtocontents{toc}{\protect\newpage}
\chapter{Система з однією аксіомою}

   \paragraph{}
   Мова програмування Ом -- це мова з залежними типами, яка є розширенням
   числення конструкцій (Calculus of Constructions, CoC) Тері Кокуанда. Саме з числення
   конструкцій починається сучасна обчислювальна математика. В додаток до CoC,
   наша мова Ом має предикативну ієрархію індексованих всесвітів. В цій мові немає
   аксіоми рекурсії для безпосереднього визначення рекурсивних типів. Однак в цій мові
   вцілому, рекурсивні дерева та корекурсія може бути визначена, або як кажуть, закодована.
   Така система аксіом називається системою з однією аксіомою (або чистою системою), тому що в ній
   існує тільки Пі-тип, а для кожного типу в теорії типів Мартіна Льофа існує чотири
   конструкції: формація, інтро, елімінатор та редуктор.

   \paragraph{}
   Усі терми підчиняються системі аксіом $Axioms$ всередині послідовності всесвітів $Sorts$
   та складність залежного терму відповідає максимальній складності домена та кодомена (правила $Rules$).
   Таким чином визначається простір всесвітів, та його конфігурація може бути записана
   згідно нотації Барендрехта для систем з чистими типами:

$$
\begin{cases}
    Sorts = Type.\{i\},\ i : Nat\\
    Axioms = Type.\{i\} : Type.\{inc\ i\}\\
    Rules = Type.\{i\} \leadsto Type.\{j\} : Type.\{max\ i\ j\}\\
\end{cases}
$$

\newpage
   \paragraph{}
   An intermediate Om language is based on Henk\cite{henk} languages described first
   by Erik Meijer and Simon Peyton Jones in 1997. Leter on in 2015 Morte impementation
   of Henk design appeared in Haskell, using Boem-Berrarducci encoding of non-recursive lambda terms.
   It is based only on one type constructor $\Pi$, its special case $\lambda$ and theirs eliminators:
   $apply$ and $curry$, infinity number of universes,
   and one computation rule called $\beta$-reduction.
   The design of Om language resemble Henk and Morte both
   design and implementation. This language indended to be small, concise, easy provable
   and able to produce verifiable peace of code that can be distributed over the networks,
   compiled at target with safe trusted linkage.

   \section{Синтаксис}
\vspace{0.5cm}
   Om syntax is compatible with $\lambda C$ Coquand's Calculus of Constructions presented
   in Morte and Henk languages. However it has extension in a part of specifying
   universe index as a {\bf Nat} number.

\vspace{0.5cm}
\begin{lstlisting}[mathescape=true]
   <> ::= #option
    I ::= #identifier
    U ::= * < #number >
    O ::= U
        | I | ( O ) | O O | O $\rightarrow$ O
        | $\lambda$ ( I : O ) $\rightarrow$ O
        | $\forall$ ( I : O ) $\rightarrow$ O
\end{lstlisting}

Equivalent tree encoding for parsed terms is following:
\vspace{0.5cm}
\begin{lstlisting}[mathescape=true]
    data name = list nat

    data om = star (n: nat)
            | var (n: name)
            | app (f a: om)
            | lambda (x: name) (d c: om)
            | arrow (d c: om)
            | pi (x: name) (d c: om)
\end{lstlisting}

\newpage
\section{Всесвіти}

The OM language is a higher-order dependently typed lambda calculus,
an extension of Coquand's Calculus of Constructions
with the predicative/impredicative hierarchy of indexed universes.
This extension is motivated avoiding paradoxes in dependent theory.
Also there is no fixpoint axiom needed for the definition
of infinity term dependance.

\vspace{0.5cm}
\begin{lstlisting}[mathescape=true]
    $U_0$ : $U_1$ : $U_2$ : $U_3$ : ...

    $U_0$ --- propositions
    $U_1$ --- values and sets
    $U_2$ --- types
    $U_3$ --- sorts
\end{lstlisting}

\begin{equation}
\tag{S}
\dfrac
{o : Nat}
{Type_o}
\end{equation}

\section{Предикативні всесвіти}

All terms obey the A ranking inside the sequence of S universes,
and the complexity R of the dependent term is equal to a maximum of
the term's complexity and its dependency.
The universes system is completely described by the following
PTS notation (due to Barendregt):

\begin{fullwidth}[width=\linewidth+4cm]
\hspace{-4cm}
\begin{lstlisting}[mathescape=true]
    $S$     (n : nat) = U n
    $A_1$ (n m : nat) = U n : U m where m > n      - cumulative
    $R_1$ (m n : nat) = U m $\rightarrow$ U n : U (max m n)  - predicative
\end{lstlisting}
\end{fullwidth}

Note that predicative universes are incompatible with Church lambda term encoding.
You can switch predicative vs impredicative uninverses by typecheker parameter.

\[
\tag{$A_1$}
\dfrac{i: Nat, j: Nat, i < j}{Type_i : Type_j}
\]

\[
\tag{$R_1$}
\dfrac{i : Nat, j : Nat}{Type_i \rightarrow Type_j : Type_{max(i,j)} }
\]

\newpage
\section{Імпредикативні всесвіти}
Propositional contractible bottom space is the only
available extension to predicative hierarchy that not leads to inconsistency.
However there is another option to have infinite
impredicative hierarchy.

\vspace{0.5cm}
\begin{lstlisting}[mathescape=true]
    $A_2$   (n : nat) = U n : U (n + 1)   - non-cumulative
    $R_2$ (m n : nat) = U m $\rightarrow$ U n : U n - impredicative
\end{lstlisting}

\begin{equation}
\tag{$A_2$}
\dfrac
{i: Nat}
{Type_i : Type_{i+1}}
\end{equation}

\begin{equation}
\tag{$R_2$}
\dfrac
{i : Nat,\ \ \ \ j : Nat}
{Type_i \rightarrow Type_{j} : Type_{j}}
\end{equation}

\section{Система з однією аксіомою}

This langauge is called one axiom language (or pure) as eliminator
and introduction adjoint functors inferred from type formation rule.
The only computation rule of Pi type is called beta-reduction.

\begin{lstlisting}[mathescape=true]
    $\forall$ (x: A) $\rightarrow$ B x : Type
    $\lambda$ (x: A) $\rightarrow$ b : B x
    f a : B [a/x]
    ( $\lambda$ (x: A) $\rightarrow$ b) a = b[a/x] : B[a/x]

\end{lstlisting}


                        \begin{equation}
                        \tag{$\Pi$-formation}
                        \dfrac
                        {x:A \vdash B : Type}
                        {\Pi\ (x:A) \rightarrow B : Type}
                        \end{equation}

                        \begin{equation}
                        \tag{$\lambda$-intro}
                        \dfrac
                        {x:A \vdash b : B}
                        {\lambda\ (x:A) \rightarrow b : \Pi\ (x: A) \rightarrow B }
                        \end{equation}

                        \begin{equation}
                        \tag{$App$-elimination}
                        \dfrac
                        {f: (\Pi\ (x:A) \rightarrow B)\ \ \ a: A}
                        {f\ a : B\ [a/x]}
                        \end{equation}

                        \begin{equation}
                        \tag{$\beta$-computation}
                        \dfrac
                        {x:A \vdash b: B\ \ \ a:A}
                        {(\lambda\ (x:A) \rightarrow b)\ a = b\ [a/x] : B\ [a/x]}
                        \end{equation}

\vspace{0.5cm}
                    This language could be embedded in itself and used
                    as Logical Framework for the Pi type:

\vspace{0.5cm}
\begin{lstlisting}[mathescape=true]

record Pi (A: Type) :=
       (intro:  (A $\rightarrow$ Type) $\rightarrow$ Type)
       (lambda: (B: A $\rightarrow$ Type) $\rightarrow$ pi A B $\rightarrow$ intro B)
       (app:    (B: A $\rightarrow$ Type) $\rightarrow$ intro B $\rightarrow$ pi A B)
       (applam: (B: A $\rightarrow$ Type) (f: pi A B) $\rightarrow$ (a: A) $\rightarrow$
                Path (B a) ((app B (lambda B f)) a) (f a))
       (lamapp: (B: A $\rightarrow$ Type) (p: intro B) $\rightarrow$
                Path (intro B) (lambda B ($\lambda$ (a:A) $\rightarrow$ app B p a)) p)

\end{lstlisting}

\newpage


   \section{Ієрархії}

\begin{lstlisting}[mathescape=true]
dep Arg Out impredicative $\rightarrow$ Out
dep Arg Out predicative   $\rightarrow$ max Arg Out

h Arg Out $\rightarrow$ dep Arg Out om:hierarchy(impredicative)
\end{lstlisting}

   \section{Перевірка всесвітів}

\begin{lstlisting}[mathescape=true]
star (:*,N) $\rightarrow$ N
star _      $\rightarrow$ (:error, "*")
\end{lstlisting}

   \section{Перевірка $\Pi$-типів}

\begin{lstlisting}[mathescape=true]
fun ((:$\forall$,),(I,O)) $\rightarrow$ true
fun T             $\rightarrow$ (:error,(:$\forall$,T))
\end{lstlisting}

   \section{Перевірка змінних}

\begin{lstlisting}[mathescape=true]
var N B        $\rightarrow$ var N B (proplists:is_defined N B)
var N B true   $\rightarrow$ true
var N B false  $\rightarrow$ (:error,("free var",N,proplists:get_keys(B)))
\end{lstlisting}

   \section{Індекси де Брейна}

\begin{lstlisting}[mathescape=true]
sh (:var,(N,I)),N,P) when I>=P $\rightarrow$ (:var,(N,I+1))
sh ((:$\forall$,(N,0)),(I,O)),N,P)      $\rightarrow$ ((:$\forall$,(N,0)),sh I N P,sh O N P+1)
sh ((:$\lambda$,(N,0)),(I,O)),N,P)      $\rightarrow$ ((:$\lambda$,(N,0)),sh I N P,sh O N P+1)
sh (Q,(L,R),N,P)               $\rightarrow$ (Q,sh L N P,sh R N P)
sh (T,N,P)                     $\rightarrow$ T
\end{lstlisting}

   \section{Нормалізація}

\begin{lstlisting}[mathescape=true]
norm :none              $\rightarrow$ :none
norm :any               $\rightarrow$ :any
norm (:app,(F,A))       $\rightarrow$ case norm F of
                            ((:$\lambda$,(N,0)),(I,O)) $\rightarrow$ norm (subst O N A)
                            NF $\rightarrow$ (:a,(NF,norm A)) end
norm (:remote,N)        $\rightarrow$ cache (norm N [])
norm (:$\rightarrow$,         (I,O)) $\rightarrow$ ((:$\forall$,("_",0)),(norm I,norm O))
norm ((:$\forall$,(N,0)),(I,O)) $\rightarrow$ ((:$\forall$,(N,0)),      (norm I,norm O))
norm ((:$\lambda$,(N,0)),(I,O)) $\rightarrow$ ((:$\lambda$,(N,0)),      (norm I,norm O))
norm T                  $\rightarrow$ T
\end{lstlisting}

\lstset{xleftmargin=0cm}

\newpage

   \section{Підстановка}

\begin{lstlisting}[mathescape=true]
sub Term Name Value          $\rightarrow$ sub Term Name Value 0
sub (:$\rightarrow$,         (I,O)) N V L $\rightarrow$ (:$\rightarrow$,           sub I N V L,sub O N V L);
sub ((:$\forall$,(N,0)),(I,O)) N V L $\rightarrow$ ((:$\forall$,(N,0)),sub I N V L,sub O N(sh V N 0)L+1)
sub ((:$\forall$,(F,X)),(I,O)) N V L $\rightarrow$ ((:$\forall$,(F,X)),sub I N V L,sub O N(sh V F 0)L)
sub ((:$\lambda$,(N,0)),(I,O)) N V L $\rightarrow$ ((:$\lambda$,(N,0)),sub I N V L,sub O N(sh V N 0)L+1)
sub ((:$\lambda$,(F,X)),(I,O)) N V L $\rightarrow$ ((:$\lambda$,(F,X)),sub I N V L,sub O N(sh V F 0)L)
sub (:app,      (F,A)) N V L $\rightarrow$ (:app,sub F N V L,sub A N V L)
sub (:var,      (N,I)) N V L when I>L $\rightarrow$ (:var,(N,I-1))
sub (:var,      (N,L)) N V L $\rightarrow$ V
sub T                  _ _ _ $\rightarrow$ T.
\end{lstlisting}


\lstset{xleftmargin=0cm}


   \section{Рівність за визначенням}

\begin{lstlisting}[mathescape=true]
eq ((:$\forall$,("_",0)), X)     (:$\rightarrow$,Y)        $\rightarrow$ eq X Y
eq (:app,(F1,A1))       (:a,(F2,A2)) $\rightarrow$ let true = eq F1 F2 in eq A1 A2
eq (:*,N)               (:*,N)       $\rightarrow$ true
eq (:var,(N,I))         (:var,(N,I)) $\rightarrow$ true
eq (:remote,N)          (:remote,N)  $\rightarrow$ true
eq ((:$\forall$,(N1,0)),(I1,O1)) ((:$\forall$,(N2,0)),(I2,O2)) $\rightarrow$
   let true = eq I1 I2 in eq O1 (sub (sh O2 N1 0) N2 (:var,(N1,0)) 0)
eq ((:$\lambda$,(N1,0)),(I1,O1)) ((:$\lambda$,(N2,0)),(I2,O2)) $\rightarrow$
   let true = eq I1 I2 in eq O1 (sub (sh O2 N1 0) N2 (:var,(N1,0)) 0)
eq (A,B) $\rightarrow$ (:error,(:eq,A,B))
\end{lstlisting}

   \section{Перевірка типів}

\begin{lstlisting}[mathescape=true]
type (:*,N)                _ $\rightarrow$ (:*,N+1)
type (:v,(N,I))            D $\rightarrow$ let true = var N D in keyget N D I
type (:#,N)                D $\rightarrow$ cache type N D
type (:$\rightarrow$,(I,O))             D $\rightarrow$ (:*,h(star(type I D)),star(type O D))
type ((:$\forall$,(N,0)),(I,O))     D $\rightarrow$ (:*,h(star(type I D)),star(type O [(N,norm I)|D]))
type ((:$\lambda$,(N,0)),(I,O))     D $\rightarrow$ let star (type I D),
                                NI = norm I in ((:$\forall$,(N,0)),(NI,type(O,[(N,NI)|D])))
type (:a,(F,A))            D $\rightarrow$ let T = type(F,D),
                                true = fun T,
                                ((:$\forall$,(N,0)),(I,O)) = T,
                                Q = type A D,
                                true = eq I Q in norm (subst O N A)
\end{lstlisting}


\section{Екстракт в платформу Erlang/OTP}

   \paragraph{}
   This works expect to compile to limited target platforms. For now Erlang, Haskell and LLVM is awaiting.
   Erlang version is expected to be useful both on LING and BEAM Erlang virtual machines.

\newpage
   \section*{Exe Macrosystem}

   Exe is a general purpose functional language with functors, lambdas on types, recursive algebraic types,
   higher order functions, corecursion, free monad for effects encoding. It compilers
   to a small core of dependent type system without recursion called Om.
   This language intended to be useful enough to encode KVS (database), N2O (web framework) and
   BPE (processes) applications.

   \section*{Compiler Passes}

   The underlying OM typechecker and compiler is a target language for EXE general purpose language.
   \begin{center}
   \begin{tabular}{ll}
   EXPAND  & EXE -- Macroexpansion\\
   NORMAL  & OM -- Term normalization and typechecking\\
   ERASE   & OM -- Delete information about types\\
   COMPACT & OM -- Term Compactification\\
   EXTRACT & OM -- Extract Erlang Code\\
   \end{tabular}
   \end{center}

   \section*{BNF}

\begin{lstlisting}[mathescape=true]
    <> ::= #option
    [] ::= #list
     I ::= #identifier
     U ::= * < #number >
     O ::= I | ( O ) |
           U | O $\rightarrow$ O | O O
             | $\lambda$ ( I : O ) $\rightarrow$ O
             | $\forall$ ( I : O ) $\rightarrow$ O
     L ::= I | L I
     A ::= O | A $\rightarrow$ A | ( L : O )
     F ::= $\empty$ | F ( I : O ) | ()
     E ::= O | E data L : A := F
             | E record L : A < extend F > := F
             | E let F in E
             | E case E [ | I O $\rightarrow$ E ]
             | E receive E [ | I O $\rightarrow$ E ]
             | E spawn E raise L := E
             | E send E to E
\end{lstlisting}

\newpage
  \section{Індуктивні типи}

  \paragraph{}
  There are two types of recursion: one is least fixed point (as $F_A\ X = 1 + A\times X$
  or $F_A\ X = A + X\times X$), in other words the recursion with a base (terminated with a bounded value),
  lists and trees are examples of such recursive structures (so we call induction recursive sums);
  and the second is greatest fixed point or recursion withour a base (as $F_A\ X = A\times X $) ---
  such kind of recursion on infinite lists (codata, streams, coinductive types) we can call recursive products.\\

  \section{Поліноміальні функтори}
  Least fixed point trees are called well-founded trees and encode polynomial functors.

  \paragraph{}
  Natural Numbers: $\mu\ X \rightarrow 1 + X$\\
  List A: $\mu\ X \rightarrow 1 + A \times X$\\
  Lambda calculus: $\mu\ X \rightarrow 1 + X \times X + X$\\
  Stream: $\nu\ X \rightarrow A \times X$\\
  Potentialy Infinite List A: $\nu\ X \rightarrow 1 + A \times X$\\
  Finite Tree: $\mu\ X \rightarrow \mu\ Y \rightarrow 1 + X \times Y = \mu\ X = List\ X$\\

  \paragraph{}
  As we know there are several ways to appear for variable in recursive algebraic type.
  Least fixpoint are known as an recursive expressions that have a base of recursion
  Both recursive and corecursive datatypes could be encoded using Boem-Berarducci encoding
  as an non-recursive definitions of folds that include in indentity signature all the
  constructor components of (co)inductive type.
\newpage
  \section{Кодування List}
  The data type of lists over a given set A can be represented as the initial algebra
  $(\mu L_A, in)$ of the functor $L_A(X) = 1 + (A \times X)$. Denote $\mu L_A = List(A)$.
  The constructor functions $nil: 1 \rightarrow List(A)$ and
  $cons: A \times List(A) \rightarrow List(A)$ are defined by
  $nil = in \circ inl$ and $cons = in \circ inr$, so $in = [nil,cons]$.
  Given any two functions $c: 1 \rightarrow C$ and $h: A \times C \rightarrow C$,
  the catamorphism $f = \llparenthesis [c,h] \rrparenthesis : List(A) \rightarrow C$
  is the unique solution of the equation system:
\vspace{0.3cm}
$$
\begin{cases}
  f \circ nil  = c \\
  f \circ cons = h \circ (id \times f)
\end{cases}
$$

\paragraph{}
  where $f = foldr(c,h)$. Having this the initial algebra is presented with functor
  $\mu (1 + A \times X)$ and morphisms sum $[1 \rightarrow List(A), A \times List(A) \rightarrow List(A)]$
  as catamorphism. Using this encdoding the base library of List will have following form:

\vspace{0.5cm}
$$
\begin{cases}
 foldr = \llparenthesis [ f \circ nil , h] \rrparenthesis, f \circ cons = h \circ (id \times f)\\
 len = \llparenthesis [ zero, \lambda\ a\ n \rightarrow succ\ n ] \rrparenthesis \\
 (++) = \lambda\ xs\ ys \rightarrow \llparenthesis [ \lambda (x) \rightarrow ys, cons ] \rrparenthesis (xs) \\
 map = \lambda\ f \rightarrow \llparenthesis [ nil, cons \circ (f \times id)] \rrparenthesis
\end{cases}
$$

\begin{lstlisting}[mathescape=true]
             data list: (A: *) $\rightarrow$ * :=
                  (nil: list A)
                  (cons: A $\rightarrow$ list A $\rightarrow$ list A)
\end{lstlisting}
$$
\begin{cases}
list = \lambda\ ctor \rightarrow \lambda\ cons \rightarrow \lambda\ nil \rightarrow ctor\\
cons = \lambda\ x\ \rightarrow \lambda\ xs \rightarrow \lambda\ list \rightarrow \lambda\ cons \rightarrow\ \lambda\ nil \rightarrow cons\ x\ (xs\ list\ cons\ nil)\\
nil = \lambda\ list \rightarrow \lambda\ cons \rightarrow \lambda\ nil \rightarrow nil\\
\end{cases}
$$
\begin{lstlisting}[mathescape=true]
           record lists: (A B: *) :=
                  (len: list A $\rightarrow$ integer)
                  ((++): list A $\rightarrow$ list A $\rightarrow$ list A)
                  (map: (A $\rightarrow$ B) $\rightarrow$ (list A $\rightarrow$ list B))
                  (filter: (A $\rightarrow$ bool) $\rightarrow$ (list A $\rightarrow$ list A))
\end{lstlisting}
$$
\begin{cases}
len = foldr\ (\lambda\ x\ n \rightarrow succ\ n)\ 0\\
(++) = \lambda\ ys \rightarrow foldr\ cons\ ys\\
map = \lambda\ f \rightarrow foldr\ (\lambda x\ xs \rightarrow cons\ (f\ x)\ xs)\ nil\\
filter = \lambda\ p \rightarrow foldr\ (\lambda x\ xs \rightarrow if\ p\ x\ then\ cons\ x\ xs\ else\ xs)\ nil\\
foldl = \lambda\ f\ v\ xs = foldr\ (\lambda\ xg\rightarrow\ (\lambda \rightarrow g\ (f\ a\ x)))\ id\ xs\ v\\
\end{cases}
$$

\vspace{1cm}
\section{Нормальні форми}

\subsection*{Lists/Map}
{\fontfamily{pcr}\selectfont
\vspace{0.5cm}
$\lambda$ (a: *) $\rightarrow$ $\lambda$ (b: *) $\rightarrow$ $\lambda$ (f: a $\rightarrow$ b) $\rightarrow$ $\lambda$ (xs: $\forall$ (List: *)
$\rightarrow$ $\forall$ (Cons: $\forall$ (head: a) $\rightarrow$ $\forall$ (tail: List) $\rightarrow$ List) $\rightarrow$ $\forall$ (Nil: List) $\rightarrow$ List)
$\rightarrow$ xs ($\forall$ (List: *) $\rightarrow$ $\forall$ (Cons: $\forall$ (head: b) $\rightarrow$ $\forall$ (tail: List) $\rightarrow$ List)
$\rightarrow$ $\forall$ (Nil: List) $\rightarrow$ List) ($\lambda$ (head: a) $\rightarrow$ $\lambda$ (tail: $\forall$ (List: *) $\rightarrow$
$\forall$ (Cons: $\forall$ (head: b) $\rightarrow$ $\forall$ (tail: List) $\rightarrow$ List) $\rightarrow$ $\forall$ (Nil: List)
$\rightarrow$ List) $\rightarrow$ $\lambda$ (List: *) $\rightarrow$ $\lambda$ (Cons: $\forall$ (head: b) $\rightarrow$ $\forall$
(tail: List) $\rightarrow$ List) $\rightarrow$ $\lambda$ (Nil: List) $\rightarrow$ Cons (f head) (tail List Cons Nil))
($\lambda$ (List: *) $\rightarrow$ $\lambda$ (Cons: $\forall$ (head: b) $\rightarrow$ $\forall$ (tail: List) $\rightarrow$
List) $\rightarrow$ $\lambda$ (Nil: List) $\rightarrow$ Nil)
}
